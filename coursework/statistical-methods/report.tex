\documentclass{article}
\usepackage{geometry}
\title{COMP80131 Assignment 2\\Statistical Methods}
\author{Michael Rawson}
\begin{document}
\maketitle
\section{Corpulence}
I assume ``uncertainty'' to mean precisely the absolute possible error in the Corpulence Index.
Define
\[
C(m, h) = \frac{m}{h^3}
\]
Then the absolute error \(\Delta C\) is given by
\[
\Delta C(m, h) = \sqrt{\left(\frac{\partial C}{\partial m}\Delta m\right)^2 + \left(\frac{\partial C}{\partial h}\Delta h\right)^2}
\]
which simplifies to
\[
\Delta C(m, h) = \frac{\sqrt{h^2\Delta m^2 + 9m^2\Delta h^2}}{h^4}
\]
Since \(\Delta m, \Delta h\) are known, this may be used to compute an absolute error for a given mass and height.
For individual 1, they have a height of 1.6736m, and a mass of 62.51kg.
Their precise Corpulence Index without allowing for error is 13.335045361004468\ldots and the error in the measurement is 0.7183773671149155\ldots --- therefore, the Corpulence Index may be given as 13.3 \(\pm\) 0.7.

The second sentence is ambiguous.
I interpret it to mean ``compute the mean, variance and uncertainty in the mean for each of height, mass and CI''.
Uncertainty in the mean is given by \(\frac{\sigma}{\sqrt{N}}\), where \(\sigma^2\) is the variance.
\begin{center}
\begin{tabular}{l c c c}
& Mean & Variance & Uncertainty in the Mean \\
\hline
Height (m) & 1.775 & 0.003108 & 0.00544\\
Mass (kg) & 80.34 & 132.2 & 1.12\\
CI (kg\(m^{-3}\)) & 14.46 & 6.489 & 0.249\\
\end{tabular}
\end{center}

\section{Clinical Trial}
The total number of patients in each case is 49.
With no treatment, 39 patients recovered, compared to 41 with a placebo, and 43 of those treated.
These data suggest that the ``placebo effect'' has some effect in treating this illness, and that the treatment also has an effect, greater than that of the placebo --- however, these results may not be statistically significant.
Additionally, the data also suggests that the majority (nearly 80\%) will recover without intervention.

I compute the probability that recovering with a placebo and recovering with the treatment are equally likely --- if this is high, the data does not \emph{strongly} suggest that the treatment is more effective than the placebo.
To do this, I make some assumptions: I assume that testing a series of individuals for recovery (either from the placebo or the treatment set) forms a Bernoulli process, and hence that the number of recoveries in a sample of size \(N\) (e.g. 49) is binomially-distributed with a prior probability \(p\).
Hence we can apply Fisher's exact test (\(\chi^2\) could also be appropriate) to the following contingency table:

\begin{center}
\begin{tabular}{l c c}
& Placebo & Treatment\\
\hline
Recovered & 41 & 43\\
Did Not Recover & 8 & 6\\
\end{tabular}
\end{center}

The exact test returns \(p = 0.7738\), which does not imply at any significance level that the treatment has a different probability of success than the placebo.

From this analysis, I would recommend gathering more data if possible in order to try and establish a more-significant marginal advantage for the treatment.
It also appears medically suspect that there is a binary ``recovery'' variable, especially considering the demise of those untreated --- this is a very severe ``illness'', on par with the plague! Those who do not perish are unlikely to have recovered fully, so I would recommend recording more nuanced data about patients.

\section{Algorithms}
First, I compute both the mean and the standard deviation for all the algorithms presented.
The author's algorithm has the quickest average runtime over the 500 samples, 44.28s, with a standard deviation of 17.62.
However, the closest competitor (State-of-the-Art 1) is not much slower at 48.13s, with a standard deviation of 17.32, slightly less variable.
None of the other algorithms are as fast or as consistent as either of these.

To establish whether the author's algorithm performance is \emph{significantly} different (we have already shown it to be faster in this dataset) than the competitor (assuming that the runs are representative of algorithm ``performance'', and that the samples follow a normal distribution), I perform a two-sample \(t\)-test to establish a \(p\)-value for this hypothesis.
Using the formulae for standard error, \(t\), and degrees of freedom from lectures, the standard error is 1.105, \(t = -3.48\), and there are 997 degrees of freedom.
Converting this to a \(p\)-value, I obtain \(p = 0.0002\), a highly-significant result.

It should be noted that whether this is of practical significance depends greatly on the target domain.
\end{document}
