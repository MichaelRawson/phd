\newcommand{\tick}{\ding{52}}

%
% Pieces from the old manual that are still useful for the future.
%

\newcommand{\old}[1]{}
%\newcommand{\old}[1]{{\color{red}#1}}

%
% Re-definitions of LaTeX commands
%

\renewcommand{\floatpagefraction}{0.9}

% part (to produce a correct entry in TOC
%\makeatletter
%\def\@part[#1]#2{%
%    \ifnum \c@secnumdepth >-2\relax
%      \refstepcounter{part}%
%%      \addcontentsline{toc}{part}{\thepart\hspace{1em}#1}%
%      \addcontentsline{toc}{part}{#1}%
%    \else
%      \addcontentsline{toc}{part}{#1}%
%    \fi
%    \markboth{}{}%
%    {\centering
%     \interlinepenalty \@M
%     \normalfont
%     \ifnum \c@secnumdepth >-2\relax
%       \huge\bfseries \partname~\thepart
%       \par
%       \vskip 20\p@
%     \fi
%     \Huge \bfseries #2\par}%
%    \@endpart}
%\makeatother

\newcommand{\fullref}[1]{\ref{#1} on page~\pageref{#1}}

%
%
%   the beast
%
%

\newcommand{\Vampire}{\textsc{Vampire}}
\newcommand{\VWP}{\url{http://www.vprover.org}}

%
%   general
%

%\newcommand{\eqref}[1]{(\ref{#1})}
\newcommand{\setof}[1]{\{#1\}}
\newcommand{\xone}[2]{#1_1,\ldots,#1_{#2}}
\newcommand{\xzero}[2]{#1_0,\ldots,#1_{#2}}
\newcommand{\QEDsymbol}{\text{\ding{111}}}
\newcommand{\QED}{\hfill\QEDsymbol}
\newcommand{\bydef}{\stackrel{\text{def}}{=}}       % by definition


%
%
%     Indexing
%
%

\newcommand{\dindex}[1]{\index{#1|textbf}}
\newcommand{\D}[1]{\DI{#1}{#1}}                       % emphasize + boldface index
\newcommand{\DI}[2]{\emph{#1}\dindex{#2}}             % emphasize + boldface index
\newcommand{\DII}[3]{\emph{#1}\dindex{#2}\dindex{#3}} % emphasize + 2 boldface indices
\newcommand{\DIII}[4]{\emph{#1}\dindex{#2}\dindex{#3}\dindex{#4}} % emphasize + 3 boldface indices
\newcommand{\DIIII}[5]{\emph{#1}\dindex{#2}\dindex{#3}\dindex{#4}\dindex{#5}} % emphasize + 4 boldface indices
\newcommand{\E}[1]{\EI{#1}{#1}}                       % emphasize + index
\newcommand{\EI}[2]{\emph{#1}\index{#2}}     % emphasize + index
\newcommand{\EII}[3]{\emph{#1}\index{#2}\index{#3}} % emphasize + 2 indices

%
%
% Algorithms
%
%

\newcommand{\inc}{~~~~\= \+ \kill}    % used in algorithms
\newcommand{\dec}{\- \kill}         % used in algorithms

\newcommand{\reserved}[1]{\textbf{\underline{#1}}} % reserved words in algorithms
\newcommand{\semicol}{;}                  % semicolon in algorithms
\newcommand{\assign}{\texttt{:=}}                  % assignment in algorithms
\newcommand{\commentinalg}[1]{\texttt{(* #1 *)}}         % comment in algorithms
\newcommand{\PROCEDURE}{\reserved{procedure}}
\newcommand{\SUBPROCEDURE}{\reserved{subprocedure}}
\newcommand{\PARAMETERS}{\reserved{parameters}}
\newcommand{\INPUT}{\reserved{input}}
\newcommand{\OUTPUT}{\reserved{output}}
\newcommand{\IF}{\reserved{if}}
\newcommand{\VAR}{\reserved{var}}
\newcommand{\CASE}{\reserved{case}}
\newcommand{\OF}{\reserved{of}}
\newcommand{\DO}{\reserved{do}}
\newcommand{\OD}{\reserved{od}}
\newcommand{\THEN}{\reserved{then}}
\newcommand{\ELSE}{\reserved{else}}
\newcommand{\WHILE}{\reserved{while}}
\newcommand{\BEGIN}{\reserved{begin}}
\newcommand{\END}{\reserved{end}}
\newcommand{\LET}{\reserved{let}}
\newcommand{\FORALL}{\reserved{forall}}
\newcommand{\ASS}{\texttt{ := }}
\newcommand{\RETURN}{\reserved{return}}
\newcommand{\REPEAT}{\reserved{repeat}}
\newcommand{\LOOP}{\reserved{loop}}
\newcommand{\FOREACH}{\reserved{foreach}}

% saturation

\newcommand{\supS}{\mathrm{Sup}}
\newcommand{\isS}{\mathbb{I}}
\newcommand{\deleted}{\cancel}
\newcommand{\Mark}{\ensuremath{\checkmark}}
\newcommand{\Active}{\ensuremath{\mathit{active}}}
\newcommand{\DeActive}{\mathit{deactivated}}
\newcommand{\Kept}{\mathit{kept}}
\newcommand{\Passive}{\ensuremath{\mathit{passive}}}
\newcommand{\Init}{\mathit{init}}
\newcommand{\New}{\mathit{new}}
\newcommand{\Given}{\mathit{given}}
\newcommand{\Other}{\mathit{other}}
\newcommand{\Select}{\mathit{select}}
\newcommand{\FInfer}{\mathit{forward\_infer}}
\newcommand{\BInfer}{\mathit{backward\_infer}}
\newcommand{\Unprocessed}{\ensuremath{\mathit{unprocessed}}}
\newcommand{\Locked}{\mathit{locked}}
\newcommand{\Children}{\ensuremath{\mathit{children}}}
\newcommand{\Reduced}{\ensuremath{\mathit{reduced}}}
\newcommand{\Queue}{\ensuremath{\mathit{sat\_queue}}}
\newcommand{\Pop}{\mathit{pop}}
\newcommand{\Simplify}{\mathit{simplify}}
\newcommand{\Retained}{\mathit{retained}}
\newcommand{\ForwardSimplify}{\mathit{forward\_simplify}}
\newcommand{\Process}{\mathit{process}}
\newcommand{\BackwardSimplify}{\mathit{backward\_simplify}}
\newcommand{\GoalFound}{\mathit{goal\_found}}
\newcommand{\Int}{\mathit{model}}        % the SAT interpretation
\newcommand{\NewInt}{\mathit{new\_model}}        % the SAT interpretation
\newcommand{\Clauses}{\mathit{cls}}    % the set of clauses stored by SAT
\newcommand{\Infer}{\mathit{infer}}

%
%
%   Logical notation
%
%

\newcommand{\imply}{\rightarrow}
\renewcommand{\implies}{\imply}
\newcommand{\liff}{\Leftrightarrow}
\newcommand{\lniff}{\not\Leftrightarrow}
\newcommand{\iffl}{\leftrightarrow}
\newcommand{\xor}{\otimes}
\newcommand{\orl}{\vee}
\newcommand{\bigorl}{\bigvee}
\newcommand{\andl}{\wedge}
\newcommand{\bigandl}{\bigwedge}
\newcommand{\notl}{\neg}
\renewcommand{\models}{\vDash}
\newcommand{\nmodels}{\nvDash}
\newcommand{\subst}[2]{#1 \mapsto #2}       % for substitutions
\newcommand{\Subst}[1]{\{#1\}}      % for substitutions
\newcommand{\emptysubst}{\varepsilon}
\newcommand{\true}{\mathsf{true}}      % boolean value true
\newcommand{\false}{\mathsf{false}}     % boolean value false
\newcommand{\eql}{\simeq}           % equality
\newcommand{\neql}{\not\simeq}           % equality

%
%
% Drawing options nicely
%
%

\newlength{\rhtwidth}
\newlength{\lftwidth}
\settowidth{\lftwidth}{\textbf{Default value:}}
\setlength{\rhtwidth}{\textwidth}
\addtolength{\rhtwidth}{-\lftwidth}
\addtolength{\rhtwidth}{-1ex}

\newcommand{\OPTHEAD}[3]{
 \vspace{1em}
  \noindent\hrulefill
  \hspace{1em}\large{\textbf{\textit{#1} options}}\phantomsection\label{optionsec:#2}\hspace{1em}
  \hrulefill\\
  \noindent
  #3
   \vspace{1em}
  }


\newenvironment{OPT}[3]{%
\minipage{\linewidth}
  \renewcommand{\tabcolsep}{0pt}%
  \renewcommand{\arraystretch}{1.35}%
  \noindent\hrulefill\\
  \textbf{\large #1}\dindex{#1@\texttt{--#1}} \hfill \emph{#2}\phantomsection\label{opt:#3}\\*[-1ex]%
  \noindent\null\hrulefill\medskip\par
  \noindent%
  \begin{tabular}{r@{\hspace*{1ex}}p{\rhtwidth}}
}{%
\end{tabular}
\medskip
\endminipage
}

\newlength{\crhtwidth}
\newlength{\clftwidth}
\settowidth{\clftwidth}{\textbf{Values:}}
\setlength{\crhtwidth}{\textwidth}
\addtolength{\crhtwidth}{-\clftwidth}
\addtolength{\crhtwidth}{-1ex}

\newenvironment{COM}[1]{%
  \renewcommand{\tabcolsep}{0pt}%
  \renewcommand{\arraystretch}{1.35}%
  \noindent\hrulefill\\
  \textbf{\large #1}\dindex{#1@\texttt{--#1}}\\*[-1ex]%
  \noindent\null\hrulefill\medskip\par
  \noindent%
  \begin{tabular}{r@{\hspace*{1ex}}p{\crhtwidth}}
}{%
\end{tabular}
\medskip
}

\newcommand{\explanation}[2]{\textbf{#1:} & #2}
\newcommand{\short}[1]{\explanation{Short form}{\optionI{#1}}}

% XMl command
\newenvironment{XMLCOM}[1]{%
  \renewcommand{\tabcolsep}{0pt}%
  \renewcommand{\arraystretch}{1.35}%
  \noindent\hrulefill\\
  \texttt{\large #1}\\*[-1ex]%
  \noindent\null\hrulefill\medskip\par
  \noindent%
  \begin{tabular}{r@{\hspace*{1ex}}p{\rhtwidth}}
}{%
\end{tabular}
\medskip
}

\newenvironment{vinput}{\noindent\begin{small}\begin{tt}\begin{tabbing}}{\end{tabbing}\end{tt}\end{small}}

% AVATAR notation

\newcommand{\FO}{\textsf{FO}} % superposition prover
\newcommand{\SAT}{\textsf{SAT}} % SAT solver
\newcommand{\assert}{\texttt{assert}} % assert command
\newcommand{\retract}{\texttt{retract}} % retract command
\newcommand{\lock}{\mathit{lock}} % lock function
\newcommand{\aclause}[2]{(#1 \leftarrow #2)}
\newcommand{\DPLLG}{\mathrm{DPLL(\Gamma)}} % DPLL(G) calculus

%
%
%   Local notation
%
%

\newcommand{\vampNot}{\~{}}    % negation in Vampire's output
\newcommand{\opt}[1]{\texttt{#1}}
\newcommand{\option}[1]{\opt{-#1}}
\newcommand{\Option}[1]{\opt{--#1}}
\newcommand{\OptionPR}[1]{\opt{--#1} on page~\pageref{opt:#1}}
\newcommand{\optionval}[2]{\texttt{-#1~#2}}
\newcommand{\Optionval}[2]{\texttt{--#1~#2}}
\newcommand{\optionI}[1]{\texttt{-#1}\index{#1@\texttt{-#1}}}
\newcommand{\OptionI}[1]{\texttt{--#1}\index{#1@\texttt{--#1}}}
\newcommand{\optionvalI}[2]{\texttt{-#1~#2}\index{#1@\texttt{-#1}}}
\newcommand{\OptionvalI}[2]{\texttt{--#1~#2}\index{#1@\texttt{--#1}}}
\newcommand{\stat}[1]{\texttt{#1}\index{#1@\texttt{#1} (statistics)}}

\newcommand{\emptyclause}{\Box}
\newcommand{\ruleBR}{\mathsf{BR}}
\newcommand{\ruleFact}{\mathsf{Fact}}
\newcommand{\nat}{\mathbb{N}}     % set of natural numbers
\newcommand{\KBo}{\succ}
\newcommand{\KBoeq}{\succeq}
\newcommand{\rr}{~\Rightarrow~}   % rewrite

%
% orderings
%
\newcommand{\occ}[2]{\mathit{occ}_#1(#2)} % number of occurrences of #1 in #2
\newcommand{\weight}{\mathit{weight}}     % weight function	
\newcommand{\level}{\mathit{level}}       % level function	
\newcommand{\precedence}{\mathit{prec}}   % precedence function	

%
% Appendix on parameters
%
\newcommand{\oref}[1]{\pageref{opt:#1}}   % reference to option description

