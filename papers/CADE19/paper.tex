\documentclass{llncs}
\title{Older or heavier? Decaying gracefully with age/weight shapes.}
\titlerunning{Dynamic Strategy Priority}
\authorrunning{Reger \and Rawson}
\author{Michael Rawson \and Giles Reger}
\institute{University of Manchester, Manchester, UK}

\begin{document}
\maketitle
\begin{abstract}
Vampire~\cite{vampire} is an automatic theorem prover for first-order logic.
During proof search in given-clause algorithms, Vampire repeatedly selects clauses from either an \emph{age} or a \emph{weight} queue in a fixed, but configurable \emph{age/weight ratio} (AWR).
We show that an optimal fixed value of this ratio can produce proofs significantly more quickly on a given problem, and further that varying AWR during proof search can improve upon a fixed ratio.
Based on this observation we develop several new modes for Vampire which vary AWR according to a ``shape'' during proof search.
The modes solve a number of new problems in the TPTP~\cite{tptp} benchmark.
\end{abstract}

\section{Introduction}
TODO: standard Vampire fluff here.\\
TODO: explain a generic given-clause algorithm, define clause age and weight.\\
TODO: remark on portfolio modes - more choice is (sometimes) a good thing.

\section{Optimising Age/Weight Ratios}
TODO: talk about optimising for fixed age/weight.\\
TODO: talk about

\section{Variable AWR for Vampire}
TODO: describe "decay" and "converge" modes for Vampire.\\
TODO: discuss the effect of the "frequency" flag, plus alternatives.

\section{Experimental Evaluation}
TODO: analyse (two sets of) results, discuss drop in performance in exchange for uniques.

\section{Future Work}
TODO: more shapes, better ways of doing frequency decay, integrate into existing strategy schedules.

\section{Conclusions}
TODO: this is a thing, you can do it, you should do it (?), we can probably do better.

\bibliographystyle{plain}
\bibliography{references}
\end{document}
