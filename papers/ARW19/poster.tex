\documentclass[24pt, a1paper, portrait]{tikzposter}
\tikzposterlatexaffectionproofoff
\usepackage{tikz}
\usepackage{amsmath}
\usepackage{enumitem}
\usetikzlibrary{shapes, arrows}
%\usepackage[sfdefault]{roboto}
\usepackage[T1]{fontenc}

\usetheme{Simple}
\definecolorstyle{manchester}
{
	\definecolor{highlight}{rgb}{0.414, 0.171, 0.566}
}
{
	\colorlet{titlebgcolor}{highlight}
	\colorlet{blocktitlefgcolor}{highlight}
}
\usecolorstyle{manchester}

\tikzstyle{decision} = [diamond, fill=highlight, text=white, align=center, node distance=15cm]
\tikzstyle{block} = [rectangle, fill=highlight, text=white, rounded corners, align=center, node distance=15cm, minimum height=4cm]
\tikzstyle{cloud} = [ellipse, fill=highlight, text=white, align=center, node distance=15cm, minimum height=4cm]
\tikzstyle{arrow} = [->, draw, draw=highlight, line width=0.3cm]

\title{A Proof Calculus for Learned Heuristics}
\author{Michael Rawson, Giles Reger}
\date{April 2018}
\institute{University of Manchester, UK}
 
\begin{document}
\maketitle

\block{Abstract}
{
	Machine learning techniques may prove capable of efficiently guiding search for proofs of human theorems.
	Many modern automated theorem provers (E, {\sc Spass}, iProver, Vampire, \ldots) make use of proof calculi such as saturation or instance generation which, while efficient, are unsuitable for current machine-learning technology.
	We propose the use of a type-theoretic calculus for a future theorem prover which attempts to leverage machine-learning techniques.
}

\begin{columns}
\column{0.5}
\block{The Prover Cycle}
{
	\begin{tikzfigure}[A generalisation of information flow in a modern prover. Note the impact that a strong (learning) heuristic may have.]
		\begin{tikzpicture}
			\node [decision] (choose) {choose next\\step};
			\node [cloud, right of=choose] (heuristic) {heuristic};
			\node [block, below of=choose] (perform) {execute step};
			\node [block, right of=perform] (process) {process\\consequences};

			\path [arrow] (choose) edge node [midway, right, align=center] {proof\\step} (perform);
			\path [arrow] (perform) edge node [midway, above] {new state} (process);
			\path [arrow] (process) edge node [sloped, above] {stable state} (choose);
			\path [arrow] (process) edge node [midway, right, align=center] {prover\\data} (heuristic);
			\path [arrow] (heuristic) edge node [midway, above] {guidance} (choose);
		\end{tikzpicture}
	\end{tikzfigure}
}

\block{Existing Calculi: Problems for ML}
{
	Existing calculi such as resolution do not adapt well to current machine-learning methods, for several reasons:

	\begin{itemize}[leftmargin=1.5cm]
		\item Artificial pre-processing (CNF, Skolemisation) reduces human structure in proof state
		\item Large, recursive input space: proof state
		\item Huge number of outputs to predict: possible next steps from an unbounded set of clauses
		\item A reliance on high performance for finding proofs
	\end{itemize}
	A better calculus would address these issues while retaining the reasonably-short proofs and high performance of existing calculi.
}

\column{0.5}
\block{Type Theory for First-Order Logic}
{
	Via the Curry-Howard correspondence (``propositions-as-types''), we can encode first-order logic in a system that resembles LF.
	For example, the theorem
	\[
		(\forall x, P(x)) \implies P(c) \wedge P(d)
	\]
	might be shown with the \(\lambda\)-term
	\[
		\lambda p : (\forall x, P(x)).\ \textrm{and}(p(c), p(d))
	\]
	Proof search is now term synthesis.
}

\block{Type Theory for Proof Search}
{
	By allowing ``holes'' in the system, search for an appropriate term can proceed iteratively: consider \emph{modus ponens}, translated into a type:
	\[
		P \to (P \to Q) \to Q
	\]

	This can obtained by repeatedly refining holes, \(?\), such that the term at any point still satisfies the target type.
	Note the convergence of the inferred type onto the target type as the proof proceeds, and the limited amount of choice at each step.

\begin{center}
	\begin{tabular}{l | l}
		\\
		Refined Term & Inferred Type \\
		\(?\) & \(?\) \\
		\(\lambda x:(P).?\) & \(P \to\ ?\)\\
		\(\lambda x:(P).\lambda f:(P \to Q).?\) & \(P \to (P \to Q) \to\ ?\)\\
		\(\lambda x:(P).\lambda f:(P \to Q). \left(?\ ?\right)\) & \(P \to (P \to Q) \to\ ?\)\\
		\(\lambda x:(P).\lambda f:(P \to Q). \left(f\ ?\right)\) & \(P \to (P \to Q) \to\ Q\)\\
		\(\lambda x:(P).\lambda f:(P \to Q). \left(f\ x\right)\) & \(P \to (P \to Q) \to\ Q\)\\
		\\
	\end{tabular}
\end{center}
	The input state is now more structured, and much smaller.
	Additionally, to fill any given hole, only a subset of all available variables will be appropriate for the target type --- this reduces the output space dramatically.
	Deductions and machine-learnt heuristics may also occur in parallel, as proof steps are not inter-dependent.
}

\end{columns}
\end{document}
